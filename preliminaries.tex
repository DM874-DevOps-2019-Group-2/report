% 2 - Preliminaries. Give a brief overview of the background knowledge needed to understand your report. Provide references to what you have used and report the main concepts that your work is based on.

\section{Preliminaries}\label{sec:preliminaries}
In this section we will provide an overview of the concepts, tools and platforms that we have used and what they are.
\subsection{Containerization}
Containerization is the concept of packaging the environment your application needs with the application itself into a small self-contained package.
In our project we have used Docker for this.
\subsection{REST}
REST is an API spec, that is built ontop of HTTP. We use REST for interacting with our service from the outside, since it is almost an ISO standard for how to expose your web-services.
\subsection{Kubernetes}
Kubernetes is a self-healing, fast, fault-tolerant, and pluggable spec, for container orchestration and networking abstraction at scale.
\subsubsection{Self-healing}
In the context of kubernetes, self-healing means that the kubernets master actively makes sure that containers are healthy, if not they are restarted (usually).
\subsubsection{Fast}
Kubernetes services are usually configured at one of the networking layers, making it extremely fast, and runtime-configurations is implemented at systemcall-level.
The whole system is also written in Go, which is known to have very good performance.
\subsubsection{Pluggable}
Kubernetes itself is more of an api that usually has pre built-in modules.
An example is the kube-dns, which can easily be plugged with another DNS.
The kubernets ingress controllers can also be plugged for something like nginx, if chosen to.
\subsubsection{Container orchestration}
Usually people know this by automatic scheduling and resource management for containers.
Imagine having 3 virtual machines and 50 containers, kubernetes automatically handles all the deploying and networking for you.
\subsection{Prometheus}
Prometheus is a applications monitoring spec, like JVM memory details, Go garbage collector details and much much more.
It is also a server software, that can pull from REST endpoints (usually /prometheus), and then exposes another api with all the collected data that can be consumed by eg Grafana.
\subsection{Grafana}
Grafana is a visualization tool, it can be connected to various sources (such as prometheus and loki), for data visualization.
\subsection{Loki}
Loki is a brand new software from the grafana team, that harvests logs stores them in cassandra and tokenizes them for extremely fast lookup.
Loki exposes an endpoint that adheres to the prometheus spec, so it is also extremely pluggable.
\subsection{Redis}
Redis is an extremely high-performance in-memory simple key-value store but extensible through an plugin environment.
Redis is used often for query caching and session management.
\subsection{Postgres}
Postgres is a state of the art high performance RDBMS.
\subsection{Kafka}
Kafka is a fully distributed, pub-sub like messaging system that employs log techniques instead of the ordinary ack-pop of pub-sub systems.
Kafka is the most widely used modern messaging system of it's category of pub-sub systems.
\subsection{JSON}
JSON is a data modelling language, it is simple and easy to read.
\subsection{JWT}
JWT stands for Json-Web-Token, which is a method of dispensing authentication tokens to users.
\subsection{GCS}
GCS stands for google cloud storage, its a thirdparty object storage service from google cloud.